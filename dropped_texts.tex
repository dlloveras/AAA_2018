
Aplicada a imágenes de luz blanca, los resultados de densidad electrónica son de naturaleza absoluta, mientras que utilizando datos en extremo ultravioleta (EUV) los resultados dependen de la abundancia de hierro coronal.

Este es un primer paso hacia el desarrollo de una técnica de tomografía multi-instrumental. Esta técnica tendrá por objetivo el reconstruir la distribución 3D de diversos parámetros coronales en forma simultánea, a través del análisis conjunto de resultados tomográficos basados en datos provistos por diversos instrumentos, incluyendo coronógrafos de luz blanca, telescopios EUV y coronógrafos de lineas de emisión coronal en el rango visible.
 

Applied to white light data, density results are of an absolute nature, while applied to extreme ultraviolet (EUV) data they scale with the iron abundance.
  
This is a first step towards implementation of a multi-instrument tomography technique. This method will aim at simultaneously reconstructing the 3D distribution of different coronal parameters through joint analysis of tomographic results based on data provided by multiple instruments, including white-light coronagraphs, EUV telescopes and visible emission line coronagraphs.
%We study the distribution of Fe abundance in both magnetically open and closed field structures. Our effort aims at helping to determine the absolute value of the First Ionization Potential (FIP) bias, discriminating whether the FIP effect consists of an enhancement of low-FIP elements, a depletion of high-FIP elements, or a combination of both.

 
